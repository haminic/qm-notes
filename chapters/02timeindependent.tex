\chapter{สมการ Schrödinger ที่ไม่ขึ้นกับเวลา}

\section{สมการ Schrödinger ที่ไม่ขึ้นกับเวลา}

\subsection{การแยกตัวแปร}

พิจารณาการแก้สมการ Schrödinger โดยการแยกตัวแปร (โดยเช่นเคย เมื่อแก้สมการออกมาแล้วเราจะสามารถสร้างคำตอบในรูปทั่วไปได้โดยการนำชุดของคำตอบทั้งหมดมารวมกัน เพราะสมการ Schrödinger เป็นสมการเชิงเส้น และปรากฏว่าคำตอบทั้งหมดที่ได้เป็นเซตของฟังก์ชันที่\emph{สมบูรณ์}และ\emph{ตั้งฉากกัน}) โดยถ้าให้
\[
	\Psi(x, t) = \psi(x)\,\varphi(t)
\]
เมื่อแทนใน (\ref{1schrodinger}) จะได้
\begin{align*}
    i\hbar\pdv{\Psi}{t} &= -\frac{\hbar^2}{2m} \pdv[2]{\Psi}{x} + V\Psi\\
    i\hbar\psi\odv{\varphi}{t} &= -\frac{\hbar^2}{2m}\varphi\odv[2]{\psi}{x} + V\psi\varphi\\
    i\hbar\frac{1}{\varphi}\odv{\varphi}{t} &= -\frac{\hbar^2}{2m}\frac{1}{\psi}\odv[2]{\psi}{x} + V
\end{align*}
ถ้า $V$ เป็นฟังก์ชันที่ไม่ขึ้นกับเวลาจะได้ว่าฝั่งซ้ายและขวาต้องเท่ากัน สมมติเท่ากับ $E$ ก็จะได้ว่า
\begin{align*}
    i\hbar\frac{1}{\varphi}\odv{\varphi}{t} &= E\\
    \int\frac{1}{\varphi}\odif{\varphi} &= -\frac{iE}{\hbar}\int\odif{t}
\end{align*}
ก็จะได้
\begin{eqnobox}
    \varphi(t) = e^{-iEt/\hbar}
\end{eqnobox}
(โดยจะละค่าคงที่ไว้เพราะ absorb ไว้ในผลเฉลยของ $\psi$) และอีกสมการหนึ่งเราจะเรียกว่าเป็น\emph{สมการ Schrödinger ที่ไม่ขึ้นกับเวลา}: 
\begin{ieqbox}{สมการ Schrödinger ที่ไม่ขึ้นกับเวลา}
    -\frac{\hbar^2}{2m}\odv[2]{\psi}{x} + V\psi = E\psi\label{2timeindepschrodinger}
\end{ieqbox}

ฟังก์ชันคลื่นที่เป็นผลเฉลยแบบแยกตัวแปรได้
\begin{eqnobox}
    \Psi(x, t) = \psi(x)\,e^{-iEt/\hbar}
\end{eqnobox}
จะเรียกว่าเป็น\emph{สถานะนิ่ง} (\emph{stationary state}) เพราะสังเกตว่า \emph{observable} $Q(x, p)$ ใด ๆ ไม่ขึ้นกับเวลา:
\begin{align*}
    \avg{Q} &= \infint \Psi^*\hat{Q}\Psi \odif{x}\\
    &= \infint \psi^*\cancel{e^{iEt/\hbar}}\hat{Q}\psi \cancel{e^{-iEt/\hbar}} \odif{x}\\
    &= \infint \psi^*\hat{Q}\psi \odif{x}
\end{align*}
(โอเปอเรเตอร์ $\hat{Q}$ ขึ้นอยู่กับแค่ $\hat{x}$ และ $\hat{p}$ ซึ่งไม่ขึ้นกับเวลาทั้งคู่)

\subsection{Hamiltonian}

ในกลศาสตร์ดั้งเดิมเราจะเรียกพลังงานรวมของระบบ (พลังงานจลน์บวกพลังงานศักย์) ว่าเป็น \emph{Hamiltonian} $H$ โดยที่
\[
	H(x, p) = \frac{p^2}{2m} + V(x)
\]
เราจึงสร้าง\emph{โอเปอเรเตอร์ Hamiltonian} ได้ดังนี้:
\begin{defbox}{โอเปอเรเตอร์ Hamiltonian}
    \begin{equation*}
        \hat{H} \equiv \frac{\hat{p}^2}{2m} + V(x) = -\frac{\hbar^2}{2m}\pdv[2]{}{x} + V(x)
    \end{equation*}
\end{defbox}
ก็จะได้สมการ (\ref{2timeindepschrodinger}) อยู่ในรูป
\begin{eqbox}{สมการ Schrödinger ที่ไม่ขึ้นกับเวลาในรูป Hamiltonian}
    \hat{H}\psi = E\psi
\end{eqbox}

พิจารณาการหาค่าคาดหมายและส่วนเบี่ยงเบนมาตรฐานของ $H$ ด้วยฟังก์ชันคลื่นที่แยกตัวแปรได้:
\begin{equation}
    \avg{H} = \infint \psi^*\hat{H}\psi\odif{x} = \infint \psi^* E\psi\odif{x} = E \infint \abs{\psi}^2 \odif{x} = E\tag{$\star$1}\label{2energyproof1}
\end{equation}
และ
\[
	\avg{H^2} = \infint \psi^*\hat{H}\psi\odif{x} = E^2\infint\abs{\psi}^2\odif{x} = E^2
\]
ดังนั้น
\begin{equation}
    \sigma_H^2 = \avg{H^2} - \avg{H}^2 = E^2 - E^2 = 0\tag{$\star$2}\label{2energyproof2}
\end{equation}
จาก (\ref{2energyproof1}) และ (\ref{2energyproof2}) จะได้ว่าคำตอบเหล่านี้อยู่ในสภาวะที่มีพลังงานแน่นอน (definite energy)

ต่อมา เรามาดูสมบัติต่าง ๆ ของพลังงานผลเฉลยสมการ (\ref{2timeindepschrodinger})

\begin{corbox}{บทตั้ง}
    ค่าพลังงาน $E$ ที่เป็นไปได้เป็นจำนวนจริง
\end{corbox}
\begin{proof}
    สมมติ $E \equiv E_0 + i\Gamma$ จะได้ว่า
    \[
	    \Psi(x, t) = \psi(x)\,e^{-i(E_0 + i\Gamma)t/\hbar} = \psi(x)\,e^{-iE_0t/\hbar + \Gamma t/\hbar}
    \]
    ดังนั้น
    \[
	    \infint \abs{\Psi}^2\odif{x} = \infint \Psi^*\Psi\odif{x} = e^{2\Gamma t/\hbar}\infint \psi^*(x)\,\psi(x)\odif{x}
    \]
    ซึ่งต้องเท่ากับ $1$ ทุก ๆ $t$ ดังนั้น $\Gamma$ จึงต้องเท่ากับ $0$ ทำให้ $E$ ต้องเป็นจำนวนจริง ตามต้องการ
\end{proof}

\begin{corbox}{บทตั้ง}
    เราสามารถสร้าง $\psi(x)$ ที่เป็นฟังก์ชันจริงได้เสมอ ไม่ว่า $E$ จะเป็นค่าเท่าใดก็ตาม
\end{corbox}
\begin{proof}
    สมมติ $\psi$ เป็นคำตอบของสมการ (\ref{2timeindepschrodinger}) เมื่อใส่ conjugate ไปทั้งสองฝั่งจะได้ว่า $\psi^*$ ก็เป็นผลเฉลย ดังนั้น $\psi + \psi^*$ เป็นผลเฉลยด้วย (เพราะสมการ Schrödinger เป็นสมการเชิงเส้น) ซึ่งเป็นผลเฉลยจริง ตามต้องการ
\end{proof}
ดังนั้นในการนำ $\psi$ มารวมกันจะตั้งข้อสมมติเลยว่า $\psi$ เป็นฟังก์ชันจริง

\begin{corbox}{บทตั้ง}
    ค่าพลังงาน $E$ ที่เป็นไปได้จะต้องมีค่ามากกว่า $V_\txt{min}$
\end{corbox}
\begin{proof}
    สมมติว่า $E \leq V_\txt{min}$ และย้ายข้างสมการ (\ref{2timeindepschrodinger}) จะได้
    \[
	    \odv[2]{\psi}{x} = \frac{2m}{\hbar^2}\ab(V(x) - E)\,\psi
    \]
    เนื่องจาก $V(x) - E \geq V_\txt{min} - E \geq 0$ จะได้ว่าเครื่องหมายของ $\odv[2]{\psi}{x}$ และ $\psi$ จะต้องตรงกันเสมอ ต่อมาจะพิสูจน์ว่าฟังก์ชันที่มีเงื่อนไขดังกล่าวไม่สามารถ normalize ได้ 
    
    สังเกตว่าถ้ามีค่า $x_0\in\RR$ ที่ทำให้ $\psi(x_0) > 0$ และ $\odv{\psi}{x}(x_0) = m \neq 0$ ในกรณี $m > 0$ จะได้ว่า $\psi$ จะโตขึ้นเรื่อย ๆ เมื่อ $x\to\infty$ และกรณี $m < 0$ จะได้ $\psi$ จะโตขึ้นเรื่อย ๆ เมื่อ $x\to -\infty$ ซึ่งเห็นชัดว่าไม่สามารถ normalize ได้ (กรณี $\psi(x_0) < 0$ ทำคล้ายกัน) จึงเหลือแค่กรณี $\psi(x)$ เป็นฟังก์ชันคงตัว ซึ่งก็ไม่สามารถ normalize ได้เช่นกัน
\end{proof}

\subsection{ผลเฉลยรวม}

เมื่อแก้สมการ เราจะมีชุดคำตอบของค่าพลังงาน $E$ ที่ ``เป็นไปได้'' โดยให้เป็นลำดับ $(E_n)_{n=1}^\infty$ และแต่ละค่า $E_n$ ก็จะมีคำตอบ $\psi_n$ ที่คู่กับพลังงานค่านั้น สุดท้ายแล้วเมื่อนำมารวมกัน จะได้ว่า
\begin{ieqbox}{ผลเฉลยของสมการ Schrödinger ในหนึ่งมิติ}
    \Psi(x, t) = \sum_{n=1}^\infty c_n\psi_n(x)e^{-iE_nt/\hbar}\label{2schrodingersol}
\end{ieqbox}
หมายเหตุ: \emph{เราจะนำชุดของ $(\psi_n)_{n=1}^\infty$ ให้เป็นเซตที่เป็นฟังก์ชันจริงและ orthonormal ($\infint\psi_m\psi_n\odif{x} = \delta_{mn}$ เมื่อ $\delta$ คือ Kronecker delta) ซึ่งจะพิสูจน์ว่าสามารถทำได้เสมอในบทถัดไป}

พิจารณาเงื่อนไขในการ normalize $\Psi$ ที่ได้ใน (\ref{2schrodingersol}):
\begin{eqnobox}
    \infint \Psi^*\Psi\odif{x} = \sum_{m=1}^\infty\sum_{n=1}^\infty\infint c_m^*c_n\psi_m\psi_n\odif{x} = \sum_{m=1}^\infty\sum_{n=1}^\infty c_m^*c_n\delta_{mn} = \sum_{n=1}^\infty\,\abs{c_n}^2 = 1
\end{eqnobox}
และจะได้ค่าคาดหมายของพลังงาน:
\begin{eqnobox}
    \avg{H} = \infint\Psi^*\hat{H}\Psi\odif{x} = \sum_{m=1}^\infty\sum_{n=1}^\infty\infint c_m^*c_nE_n\psi_m\psi_n\odif{x} = \sum_{n=1}^\infty\,\abs{c_n}^2E_n
\end{eqnobox}
โดยขนาดสัมประสิทธิ์ $\abs{c_n}^2$ นี้มีสมบัติคือเป็นความน่าจะเป็นที่จะวัดพลังงานได้ $E_n$ (จะพิสูจน์อีกที)

\section{บ่อศักย์อนันต์}

\subsection{บ่อศักย์อนันต์}

กำหนดให้ในบริเวณหนึ่งมี
\begin{eqnobox}
    V(x) = 
    \begin{cases}
        0\quad &\text{เมื่อ $x\in\ab[0, a]$}\\
        \infty &\text{เมื่อ $x\notin\ab[0, a]$}
    \end{cases}
\end{eqnobox}
จะได้ว่าสำหรับ $x\in\ab[0, a]$:
\begin{align*}
    -\frac{\hbar^2}{2m}\odv[2]{\psi}{x} &= E\psi\\
    \odv[2]{\psi}{x} + k^2\psi &= 0
\end{align*}
เมื่อ $k = \sqrt{2mE}/\hbar$ เป็นจำนวนจริง (เพราะจากที่พิสูจน์ไป $E$ ในกรณีนี้ไม่มีทางน้อยกว่า $V_\txt{min} = 0$) ก็จะแก้ได้ว่า
\[
	\psi(x) = A\sin kx + B\cos kx
\]
โดยเรามีเงื่อนไขที่ขอบเขต $\psi(0) = 0$ และ $\psi(a) = 0$ เพราะ $\psi$ ต้องต่อเนื่อง (พิสูจน์ทีหลัง) ดังนั้นเมื่อแทน $x = 0$ จะได้ $B = 0$ และเมื่อแทน $x = a$ จะได้ $A = 0$ หรือ $ka \in \{0, \pm\pi, \pm 2\pi, \pm 3\pi, \dots\}$ แต่ $A$ และ $k$ ห้ามเป็น $0$ เพราะมิฉะนั้นฟังก์ชันคลื่น $\Psi(x, t) = \psi(x)\,e^{-iEt/\hbar}$ จะไม่สามารถ normalize ได้

อย่างน่าแปลกใจ เราได้เงื่อนไขของ $E$ ที่เป็นไปได้จากการแก้สมการ โดยเนื่องจากค่า $k$ ที่เป็นไปได้คือ
\[
	k = \frac{n\pi}{a}\qq*{,}n\in\ZZ^+
\]
(เพราะถ้า $n$ เป็นลบเราสามารถ absorb เข้าค่าคงที่ $A$ ได้และ $n\neq 0$ เพราะ $\Psi$ จะไม่สามารถ normalize ได้) จะได้ว่า $E$ ที่เป็นไปได้คือ
\begin{eqnobox}
    E = \frac{\hbar^2k^2}{2ma^2} = \frac{n^2\pi^2\hbar^2}{2ma^2}\qq*{,}n\in\ZZ^+
\end{eqnobox}
ก็จะได้
\begin{ieqbox}{พลังงานของบ่อศักย์อนันต์}
    E_n =  \frac{n^2\pi^2\hbar^2}{2ma^2}\qq*{,}n\in\ZZ^+
\end{ieqbox}
ต่อมาหาชุดของ $(\phi_n)_{n=1}^\infty$ ที่ orthonormal (orthogonal จะสมมติเลยว่าจริงแต่จะหาแค่ $A$ ที่ทำให้ $\psi$ ถูก normalize):
\[
	\int_0^a A^2\sin^2\ab(kx)\odif{x} = A^2\ab(\frac{a}{2}) = 1
\] 
ดังนั้นได้ $A = \sqrt{2/a}$ (ถ้า $A$ ติดลบจะถูก absorb ได้เมื่อหาสัมประสิทธิ์ $c_n$ อยู่ดี จึงให้เป็นบวก) ก็จะได้
\begin{ieqbox}{สถานะนิ่งของบ่อศักย์อนันต์}
    \psi_n\ab(x) = \sqrt{\frac{2}{a}}\sin\ab(\frac{n\pi}{a}x)\qq*{,}n\in\ZZ^+
\end{ieqbox}
เราจะเรียกสถานะที่มีพลังงานตำที่สุด ($n = 1$) ว่าเป็น\emph{สถานะพื้น} (\emph{ground state}) และสถานะอื้นว่า\emph{สถานะกระตุ้น} (\emph{excited state}) และเราจะสามารถหาสัมประสิทธิ์ $c_n$ สำหรับ $\Psi(x, t)$ ใด ๆ (โดยที่มีสภาวะขอบเขต $\Psi(x, 0)$) ได้โดย
\begin{eqnobox}
    c_n = \infint \Psi(x, 0)\,\psi_n(x)\odif{x} = \sqrt{\frac{2}{a}}\int_0^a \sin\ab(\frac{n\pi}{a}x)\,\Psi(x, 0)\odif{x}
\end{eqnobox}

\section{Harmonic Oscillator เชิงควอนตัม}

\subsection{Harmonic Oscillator และ Commutator}

ในกลศาาสตร์ดั้วเดิม ศักย์ของระบบ \emph{simple harmonic oscillator} กำหนดด้วย
\[
	V(x) = \frac{1}{2}kx^2
\]
หรือ
\[
	V(x) = \frac{1}{2}m\omega^2x^2
\]
ดังนั้นเราจะพิจารณาศักย์เดียวกันในกลศาาสตร์ควอนตัม:
\begin{ieqbox}{สมการ Schrödinger ของ Harmonic Oscillator}
    -\frac{\hbar^2}{2m} \odv[2]{\psi}{x} + \frac{1}{2}m\omega^2x^2\psi = \hat{H}\psi = E\psi\label{2harmonicschrodinger}
\end{ieqbox}

โดยก่อนที่เราจะแก้สมการด้านบน เราจะนิยาม \emph{commutator} ของโอเปอเรเตอร์ $\hat{A}$ และ $\hat{B}$ ดังนี้:
\begin{defbox}{ Commutator}
    \begin{equation*}
        [\hat{A}, \hat{B}] \equiv \hat{A}\hat{B} - \hat{B}\hat{A}
    \end{equation*}
\end{defbox}
คล้ายกับการวัด ``ความไม่สลับที่ได้'' ของสองโอเปอเรเตอร์ พิจารณา $[\hat{x}, \hat{p}]$:
\begin{align*}
    [\hat{x}, \hat{p}]\,f(x) &= \ab(-i\hbar x\odv{}{x} + i\hbar\odv{}{x}x)\,f(x)\\
    &= -\cancel{i\hbar x\odv{}{x}f(x)} + \cancel{i\hbar x\odv{}{x}f(x)} + i\hbar \,f(x)\\
    &= i\hbar\,f(x)
\end{align*}
ดังนั้น
\begin{eqbox}{Commutator ของตำแหน่งและโมเมนตัม}
    [\hat{x}, \hat{p}] = i\hbar
\end{eqbox}

\subsection{วิธีเชิงพีชคณิต}

พิจารณา
\[
	\hat{H} = \frac{\hat{p}^2}{2m} + \frac{1}{2}m\omega^2x^2 = \frac{1}{2m}\ab(\hat{p}^2 + \ab(m\omega x)^2)
\]
ลองพยายาม ``แยกตัวประกอบ'' โดยนิยาม
\begin{defbox}{โอเปอเรเตอร์ขั้นบันได}
    \begin{equation*}
        \hat{a}_\pm \equiv \frac{1}{\sqrt{2\hbar m\omega}}\ab(\mp i\hat{p} + m\omega x)
    \end{equation*}
\end{defbox}
(สัมประสิทธิ์ด้านหน้ามีเพื่อให้จัดรูปสวย ๆ) จะได้ว่า
\begin{align}
    \hat{a}_-\hat{a}_+ &= \frac{1}{2\hbar m\omega}\ab(\hat{p}^2 + \ab(m\omega x)^2 -im\omega[\hat{x}, \hat{p}])\notag\\
    &= \frac{1}{2\hbar m\omega}\ab(\hat{p}^2 + \ab(m\omega x)^2 +\hbar m\omega)\notag\\
    &= \frac{1}{\hbar\omega}\hat{H} + \frac{1}{2}\tag{$\circ$1}\label{2ladderproof1}
\end{align}
และในทำนองเดียวกัน
\begin{equation}
    \hat{a}_+\hat{a}_- = \frac{1}{\hbar\omega}\hat{H} - \frac{1}{2}\tag{$\circ$2}\label{2ladderproof2}
\end{equation}
จาก (\ref{2ladderproof1}) และ (\ref{2ladderproof2}) ก็จะได้
\begin{eqnobox}[]
    [\hat{a}_-, \hat{a}_+] = 1
\end{eqnobox}
และสมการ Schrödinger (\ref{2harmonicschrodinger}) จะอยู่ในรูป
\begin{eqnobox}
    \hat{H}\psi = \hbar\omega\ab(\hat{a}_\pm\hat{a}_\mp \pm \frac{1}{2})\psi = E\psi
\end{eqnobox}

พิจารณาคำตอบหนึ่งของสมการ และนำคำตอบนั้นมา operate ด้วย $\hat{a}_+$:
\begin{align*}
    \hat{H}\ab(\hat{a}_+\psi) &= \hbar\omega\ab(\hat{a}_+\hat{a}_- + \frac{1}{2})\hat{a}_+\psi = \hat{a}_+\hbar\omega\ab(\hat{a}_-\hat{a}_+ + \frac{1}{2})\psi\\
    &= \hat{a}_+\hbar\omega\ab(\hat{a}_-\hat{a}_+ - \frac{1}{2} + 1)\psi = \hat{a}_+\ab(\hat{H} + \hbar\omega)\psi\\
    &=\ab(E + \hbar\omega)\ab(\hat{a}_+\psi)
\end{align*}
ซึ่งเป็นคำตอบของสมการ (\ref{2harmonicschrodinger}) ที่มีพลังงาน $E + \hbar\omega$ และในทำนองเดียวกันจะได้ว่า
\[
	\hat{H}\ab(\hat{a}_-\psi) = \ab(E - \hbar\omega)(\hat{a}_-\psi)
\]
ดังนั้นถ้าเรามีคำตอบหนึ่งของ (\ref{2harmonicschrodinger}) เราสามารถหาคำตอบที่มีพลังงานอื่น ๆ ได้:
\begin{lawbox}{การสร้างคำตอบใหม่ของ Harmonic Oscillator}
    ถ้ามีคำตอบ $\psi$ ของสมการ (\ref{2harmonicschrodinger}) เราจะสามารถสร้างคำตอบที่มีระดับพลังงานสูงหรือต่ำกว่าได้โดย
    \begin{equation*}
        \hat{H}\ab(\hat{a}_\pm\psi) = \ab(E \pm \hbar\omega)(\hat{a}_\pm\psi)
    \end{equation*}
\end{lawbox}
เราจึงเรียกโอเปอเรเตอร์ $\hat{a}_\pm$ ว่า\emph{โอเปอเรเตอร์ขั้นบันได}

ต่อมาสังเกตว่าเราสามารถสร้างคำตอบที่มีพลังงานลดลงได้เรื่อย ๆ ซึ่งจะขัดแย้งกับที่เคยพิสูจน์ว่า $E > 0$ ดังนั้นจะต้องมี $\psi_0$ สักตัวที่เป็น ``ขั้นบันไดขั้นแรก'' ที่เมื่อใช้ \emph{lowering operator} ($\hat{a}_-$) แล้วจะได้ฟังก์ชันที่ไม่สามารถ normalize ได้:
\[
	\hat{a}_-\psi_0 = 0
\]
(สมมติไปก่อนว่าฟังก์ชันที่ไม่สามารถ normalize ได้นั้นเป็น $0$ ไปเลย เพราะการมี square integral เป็น $\infty$ ไม่จำเป็นว่าจะไม่ต้องพิจารณา) ดังนั้น
\begin{align*}
    \cancel{\frac{1}{\sqrt{2\hbar m\omega}}}\ab(\hbar\odv{}{x} + m\omega x)\psi_0 &= 0\\
    \frac{m\omega}{\hbar}\int x\odif{x} &= -\int\frac{1}{\psi_0}\odif{\psi_0} \\
    \frac{m\omega}{2\hbar} x^2 + C &= -\log\psi_0 \\
    \psi_0 &= Ae^{-\frac{m\omega}{2\hbar}x^2}
\end{align*}
จาก
\[
	1 = \infint\psi_0^2\odif{x} = A^2\infint e^{-\frac{m\omega}{\hbar}x^2}\odif{x} = A^2\sqrt{\frac{\pi\hbar}{m\omega}}
\]
จะได้ค่าคงที่ที่ทำให้ฟังก์ชันนี้ normalize คือ
\[
	A = \ab(\frac{m\omega}{\pi\hbar})^{1/4}
\]
และพิจารณาพลังงาน ($E_0$) ที่สถานะ $\psi_0$ นี้:
\[
	\hat{H}\psi_0 = \hbar\omega\ab(\hat{a}_+\hat{a}_- + \frac{1}{2})\psi_0 = \hbar\omega\hat{a}_+\cancel{\ab(\hat{a}_-\psi_0)} + \frac{1}{2}\hbar\omega\psi_0 = \ab(\frac{1}{2}\hbar\omega)\psi_0
\]
ก็จะได้พลังงานที่สถานะพื้นนี้เท่ากับ
\[
	E_0 = \frac{1}{2}\hbar\omega
\]
สรุปก็คือ
\begin{eqbox}{สถานะพื้นของ Harmonic Oscillator}
    \psi_0(x) = \ab(\frac{m\omega}{\pi\hbar})^{1/4}e^{-\frac{m\omega}{2\hbar}x^2}\qq{และ}E_0 = \frac{1}{2}\hbar\omega
\end{eqbox}
หมายเหตู: \emph{คำตอบทั้งหมดของ (\ref{2harmonicschrodinger}) จะต้องมาจากการ operate $\hat{a}_+$ บน $\psi_0$ เพราะทุก ๆ คำตอบที่เราสนใจจะต้องเกิดสถานะพื้นที่มีเงื่อนไข $\hat{a}_-\psi_0 = 0$ เสมอ}

และก็จะได้
\begin{ieqbox}[label=2harmonicenergy]{พลังงานของ Harmonic Oscillator}
    E_n = \ab(n + \frac{1}{2})\hbar\omega\qq*{,}n\in\mathbb{Z}^+_0
\end{ieqbox}

ต่อมาพิจารณาการหาสัมประสิทธิ์ $A_n$ ของ $\psi_n$ โดยก่อนอื่นจะพิสูจน์ว่า $\hat{a}_-$ และ $\hat{a}_+$ เป็น \emph{adjoint} (หรือ \emph{hermitian conjugate}) ของกันและกัน:
\begin{eqbox}{บทตั้ง}
    \infint f^*\ab(\hat{a}_\pm g)\odif{x} = \infint \ab(\hat{a}_\mp f)^* g\odif{x}
\end{eqbox}
\begin{proof}
    เนื่องจาก integration by parts จะได้
    \[
	    \infint f^*\odv{g}{x}\odif{x} = \cancel{\eval{f^*g}_{-\infty}^\infty} - \infint \ab(\odv{f}{x})^*g\odif{x} = - \infint \ab(\odv{f}{x})^*g\odif{x}
    \]
    ดังนั้น
    \begin{align*}
        \infint f^*\ab(\hat{a}_\pm g)\odif{x} &= \infint f^* \ab(\frac{1}{\sqrt{2\hbar m\omega}}\ab(\mp \hbar\odv{}{x} + m\omega x))g\odif{x}\\ 
        &= \infint \ab(\frac{1}{\sqrt{2\hbar m\omega}}\ab(\pm \hbar\odv{}{x} + m\omega x)f)^* g\odif{x}\\
        &= \infint \ab(\hat{a}_\mp f)^* g\odif{x}
    \end{align*}
    ตามต้องการ
\end{proof}

พิจารณาให้
\[
	\hat{a}_+\psi_n = c_n\psi_{n+1}
\]
จะได้ว่า
\begin{align*}
    1 = \infint \abs{\psi_{n+1}}^2\odif{x} &= \frac{1}{c_n^2} \infint \ab(\hat{a}_+\psi_n)^*\ab(\hat{a}_+\psi_n) \odif{x} \\
    &= \frac{1}{c_n^2} \infint\ab(\hat{a}_-\hat{a}_+\psi_n)^*\psi_n\odif{x}\\
    &= \frac{1}{c_n^2} \infint\ab[\ab(\frac{1}{\hbar\omega}\hat{H} + \frac{1}{2})\psi_n]^*\psi_n\odif{x}\\
    &= \frac{1}{c_n^2} \ab(\frac{1}{\hbar\omega}E_n + \frac{1}{2})\infint \abs{\psi_n}^2\odif{x}\\
    &= \frac{1}{c_n^2} \ab(\frac{1}{\hbar\omega}E_n + \frac{1}{2})
\end{align*}
ดังนั้น
\[
	c_n = \sqrt{\frac{1}{\cancel{\hbar\omega}}\ab(n+\frac{1}{2})\cancel{\hbar\omega} + \frac{1}{2}} = \sqrt{n + 1}
\]
ก็จะได้ว่า
\begin{ieqbox}[label=2stationaryrecursion]{สถานะนิ่งของ Harmonic Oscillator ในรูปโอเปอเรเตอร์ขั้นบันได}
    \psi_n(x) = \frac{1}{\sqrt{n!}}\ab(\hat{a}_+)^n\psi_0(x)\qq*{,}n\in\mathbb{Z}^+_0
\end{ieqbox}

\subsection{วิธีเชิงวิเคราะห์}

สมมติให้
\[
    \xi \equiv \sqrt{\frac{m\omega}{\hbar}}x
\]
โดย (\ref{2harmonicschrodinger}) จะได้ว่า
\begin{eqbox}{}
    \odv[2]{\psi}{\xi} = \ab(\xi^2 - K)\,\psi
\end{eqbox}
เมื่อ $K \equiv \frac{2E}{\hbar\omega}$

เราจะแก้สมการนี้โดยอาศัย power series แต่เนื่องจากโดยทั่วไปแล้ว $\psi$ จะไม่เป็นฟังก์ชัน analytic เราจึงลองพิจารณา asymptotic behavior ของ $\psi$:

เมื่อ $\xi \to \pm\infty$ จะได้ว่า
\[
    \odv[2]{\psi}{\xi} \approx \xi^2\psi
\]
ซึ่งสามารถประมาณคำตอบได้
\begin{eqnobox}
    \psi(\xi) \approx Ae^{-\xi^2/2} + Be^{\xi^2/2}
\end{eqnobox}
แต่ส่วนหลังยังไงก็ normalize ไม่ได้จึงตัดออกไป

ต่อมาเราจะกำหนดให้
\begin{eqnobox}[label=2analyticproof1]
    \psi(\xi) = h(\xi)\,e^{-\xi^2/2}
\end{eqnobox}
และหวังว่าคำตอบที่ได้จะสามารถเขียนได้เป็น power series:
\[
    h(\xi) = a_0 + a_1\xi + a_2\xi^2 + \dots
\]
เมื่อลองนำ (\ref{2analyticproof1}) ไปแทนใน (\ref{2harmonicschrodinger}) จะได้ว่า
\begin{eqbox}{}
    \odv[2]{h}{\xi} - 2\xi\odv{h}{\xi} + \ab(K-1)h = 0
\end{eqbox}
นำ power series ของ $h$ ไปแทนจากนั้นเทียบสัมประสิทธิ์ ก็จะได้
\[
    (j+1)(j+2)a_{j+2} - 2ja_j + (K-1)a_j = 0
\]
หรือ
\begin{ieqbox}{ความสัมพันธ์เวียนเกิดของสัมประสิทธิ์ Power Series ของ h}
    a_{j+2} = \frac{2j+1-K}{(j+1)(j+2)}a_j
\end{ieqbox}
สำหรับทุก $j\in\ZZ^+$ โดยจะมีค่าขอบเขตที่เลือกได้คือ $a_0$ และ $a_1$ (ก็จะเห็นได้ว่า $h$ กำลังคู่และกำลังคี่นั้น independent ต่อกันและกัน)

แต่สังเกตว่าเมื่อ $j\to\infty$ ทำให้ $a_{j+2}\approx (2/j)\,a_j$ หรือ $a_j\approx C/(j/2)!$ ดังนั้น
\[
    h(\xi) = \Theta\ab(\sum\frac{1}{(j/2)!}\xi^j) = \Theta\ab(\sum\frac{1}{j!}\xi^{2j}) = \Theta\ab(e^{\xi^2}) 
\]
ซึ่งเห็นชัดว่าจะทำให้ $\psi$ normalize ไม่ได้ ดังนั้นค่า $K$ ที่สมเหตุสมผลจะต้องทำให้ความสัมพันธ์เวียนเกิดนี้จบลงที่เลขชี้กำลังค่าหนึ่ง แต่เราสามารถเลือกได้ให้เพียงกำลังคู่หรือกำลังคี่มีจุดจบเท่านั้น จึงทำให้เราจำเป็นต้องมี $a_0 = 0$ หรือ $a_1 = 0$

โดยค่า $K$ ที่เป็นไปได้นั้นก็คือ
\[
    K = 2n+1
\]
สำหรับ $n\in\ZZ_0^+$ หรือก็จะได้ค่าพลังงานที่เป็นไปได้เช่นเดียวกับ (\ref{2harmonicenergy})

จะเห็นว่าในแต่ละค่า $n$ เราจะได้พหุนามดีกรี $n$ ที่มีกำลังคู่หรือกำลังคี่อย่างใดอย่างหนึ่งเท่านั้นสำหรับ $h$ โดยเราจะเรียกพหุนามที่ว่านี้ (เมื่อทำให้สัมประสิทธิ์พจน์สูงสุดเป็น $2^n$ ตาม convention) ว่า\emph{พหุนาม Hermite} ($H_n$) โดยมี $6$ ตัวแรกคือ:
\begin{align*}
    H_1(\xi) &= 2\xi\\
    H_2(\xi) &= 4\xi^2 - 2\\
    H_3(\xi) &= 8\xi^3 - 12\xi\\
    H_4(\xi) &= 16\xi^4 - 48\xi^2 + 12\\
    H_5(\xi) &= 32\xi^5 - 160\xi^3 + 120\xi
\end{align*}

เราจะได้คำตอบของสมการ Schrödinger คือ
\[
    \psi_n(x) = C_nH_n(\xi)e^{-\xi^2/2}
\]
หรือเมื่อ normalize ก็จะได้
\begin{ieqbox}{สถานะนิ่งของ Harmonic Oscillator ในรูปพนุนาม Hermite}
    \psi_n(x) = \ab(\frac{m\omega}{\pi\hbar})^{1/4}\frac{1}{\sqrt{2^nn!}}H_n(\xi)e^{-\xi^2/2}
\end{ieqbox}
ซึ่งถ้าลองใช้ความสัมพันธ์เวียนเกิดแบบโอเปอเรเตอร์ขั้นบันไดใน (\ref{2stationaryrecursion}) จะเห็นได้ว่า $\psi_n$ ทั้งคู่นี้เหมือนกัน




