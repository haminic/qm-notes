\chapter{สมการ Schrödinger ที่ไม่ขึ้นกับเวลา}

\section{สมการ Schrödinger ที่ไม่ขึ้นกับเวลา}

\subsection{การแยกตัวแปร}

พิจารณาการแก้สมการ Schrödinger โดยการแยกตัวแปร (โดยเช่นเคย เมื่อแก้สมการออกมาแล้วเราจะสามารถสร้างคำตอบในรูปทั่วไปได้โดยการนำชุดของคำตอบทั้งหมดมารวมกัน เพราะสมการ Schrödinger เป็นสมการเชิงเส้น) โดยถ้าให้
\[
\Psi(x, t) = \psi(x)\,\varphi(t)
\]
เมื่อแทนใน (\ref{1schrodinger}) จะได้
\begin{align*}
    i\hbar\pdv{\Psi}{t} &= -\frac{\hbar^2}{2m} \pdv[2]{\Psi}{x} + V\Psi\\
    i\hbar\psi\odv{\varphi}{t} &= -\frac{\hbar^2}{2m}\varphi\odv[2]{\psi}{x} + V\psi\varphi\\
    i\hbar\frac{1}{\varphi}\odv{\varphi}{t} &= -\frac{\hbar^2}{2m}\frac{1}{\psi}\odv[2]{\psi}{x} + V
\end{align*}
ถ้า $V$ เป็นฟังก์ชันที่ไม่ขึ้นกับเวลาจะได้ว่าฝั่งซ้ายและขวาต้องเท่ากัน สมมติเท่ากับ $E$ ก็จะได้ว่า
\begin{align}
    i\hbar\frac{1}{\varphi}\odv{\varphi}{t} &= E\notag\\
    \int\frac{1}{\varphi}\odif{\varphi} &= -\frac{iE}{\hbar}\int\odif{t}\notag\\
    \varphi(t) &= e^{-iEt/\hbar}
\end{align}
(โดยจะละค่าคงที่ไว้เพราะ absorb ไว้ในผลเฉลยของ $\psi$) และอีกสมการหนึ่งเราจะเรียกว่าเป็น\emph{สมการ Schrödinger ที่ไม่ขึ้นกับเวลา}: 
\begin{ieqbox}{สมการ Schrödinger ที่ไม่ขึ้นกับเวลา}
    -\frac{\hbar^2}{2m}\odv[2]{\psi}{x} + V\psi = E\psi\label{2timeindepschrodinger}
\end{ieqbox}

ฟังก์ชันคลื่นที่เป็นผลเฉลยแบบแยกตัวแปรได้
\begin{equation}
    \Psi(x, t) = \psi(x) e^{-iEt/\hbar}
\end{equation}
จะเรียกว่าเป็น\emph{สภาวะนิ่ง} (\emph{stationary state}) เพราะสังเกตว่า \emph{observable} $Q(x, p)$ ใด ๆ ไม่ขึ้นกับเวลา:
\begin{align*}
    \avg{Q} &= \infint \Psi^*\hat{Q}\Psi \odif{x}\\
    &= \infint \psi^*\cancel{e^{-iEt/\hbar}}\hat{Q}\psi \cancel{e^{iEt/\hbar}} \odif{x}\\
    &= \infint \psi^*\hat{Q}\psi \odif{x}
\end{align*}
(โอเปอเรเตอร์ $\hat{Q}$ ขึ้นอยู่กับแค่ $\hat{x}$ และ $\hat{p}$ ซึ่งไม่ขึ้นกับเวลาทั้งคู่)

\subsection{Hamiltonian}

ในกลศาสตร์ดั้งเดิมเราจะเรียกพลังงานรวมของระบบ (พลังงานจลน์บวกพลังงานศักย์) ว่าเป็น \emph{Hamiltonian} $H$ โดยที่
\[
H(x, p) = \frac{p^2}{2m} + V(x)
\]
เราจึงสร้าง\emph{โอเปอเรเตอร์ Hamiltonian} ได้ดังนี้:
\begin{defbox}{โอเปอเรเตอร์ Hamiltonian}
    \begin{equation}
        \hat{H} \equiv \frac{\hat{p}^2}{2m} + V(x) = -\frac{\hbar^2}{2m}\pdv[2]{}{x} + V(x)
    \end{equation}
\end{defbox}
ก็จะได้สมการ (\ref{2timeindepschrodinger}) อยู่ในรูป
\begin{eqbox}{สมการ Schrödinger ที่ไม่ขึ้นกับเวลาในรูป Hamiltonian}
    \hat{H}\psi = E\psi
\end{eqbox}

พิจารณาการหาค่าคาดหมายและส่วนเบี่ยงเบนมาตรฐานของ $H$ ด้วยฟังก์ชันคลื่นที่แยกตัวแปรได้:
\begin{equation}
    \avg{H} = \infint \psi^*\hat{H}\psi\odif{x} = \infint \psi^* E\psi\odif{x} = E \infint \abs{\psi}^2 \odif{x} = E\tag{$\star$1}\label{2energyproof1}
\end{equation}
และ
\[
\avg{H^2} = \infint \psi^*\hat{H}\psi\odif{x} = E^2\infint\abs{\psi}^2\odif{x} = E^2
\]
ดังนั้น
\begin{equation}
    \sigma_H^2 = \avg{H^2} + \avg{H}^2 = E^2 - E^2 = 0\tag{$\star$2}\label{2energyproof2}
\end{equation}
จาก (\ref{2energyproof1}) และ (\ref{2energyproof2}) จะได้ว่าคำตอบเหล่านี้อยู่ในสภาวะที่มีพลังงานแน่นอน (definite energy)