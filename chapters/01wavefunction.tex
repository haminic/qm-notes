\chapter{ฟังก์ชันคลื่น}

\section{สถิติและความน่าจะเป็น}

\subsection{ฟังก์ชันคลื่น}

ในกลศาสตร์ดั้งเดิม เราจะอธิบายอนุภาคหนึ่ง ๆ ด้วยตำแหน่งและโมเมนตัม แต่ในกลศาสตร์ควอนตัม เราจะใช้สิ่งที่เรียกว่า\emph{ฟังก์ชันคลื่น} (\emph{wavefunction}: $\Psi(x, t)$ ในหนึ่งมิติ) ซึ่งมีโคโดเมนเป็น $\CC$

การเปลี่ยนแปลงของตำแหน่งและโมเมนตัมเมื่อเวลาผ่านไปในกลศาสตร์ดั้งเดิมจะถูกอธิบายด้วยกฎของนิวตัน แต่ในกลศาสตร์ควอนตัม เราจะอธิบายวิวัฒนาการของฟังก์ชันคลื่นด้วย\emph{สมการ Shrödinger}:
\begin{lawbox}{สมการ Shrödinger}
    อนุภาคที่มีฟังก์ชันคลื่น $\Psi$ จะมีการวิวัฒนาการเป็นไปตามสมการ
    \begin{equation}
        i\hbar\pdv{\Psi}{t} = -\frac{\hbar^2}{2m}\pdv[2]{\Psi}{x} + V\Psi\label{1shrodinger}
    \end{equation}
\end{lawbox}

\subsection{การวัดตำแหน่งและการ Normalize}

ในกลศาสตร์ควอนตั้ม อนุภาคไม่ได้มีตำแหน่งที่แน่นอนเหมือนกับกลศาสตร์ดั้งเดิม แต่จะถูกอธิบายด้วยความน่าจะเป็น โดยที่มี $\abs{\Psi(x, t)} = \Psi\cdot\Psi^*$ เป็นความหนาแน่นความน่าจะเป็นในการพบอนุภาคที่ $x$ หรือก็คือ
\begin{ieqbox}{ฟังก์ชันคลื่นกับการวัดตำแหน่ง}
    \int_a^b \abs{\Psi(x, t)}^2 \odif{x} = P(a \leq x \leq b)\label{1position}
\end{ieqbox}
โดยเมื่อมีการวัดเกิดขึ้นแล้ววัดได้ตำแหน่ง $x=d$ ที่ $t=0$ ฟังก์ชันคลื่นจะ\emph{ยุบตัว} (\emph{collapse}) ให้ในครั้งถัดไป ถ้าวัดตำแหน่งของอนุภาคทันทีหลังจากการวัดครั้งแรก ก็จะวัดได้ตำแหน่งเดิม หรือก็คือจะได้ว่า $\Psi$ ใหม่จะมีปริเวณที่ $\abs{\Psi} \neq 0$ ที่เดียวคือที่ $x=d$ เป็น $\infty$ 

สังเกิตว่าจาก (\ref{1position}) ถ้าอยากให้ $\abs{\Psi(x, t)}^2$ มีความหมายเชิงสถิติ เราจะต้องทำให้การอินทิเกรตฟังก์ชันนี้ทั่วทุกบริเวณจะต้องเป็น $1$ ซึ่งเรียกว่าเป็นการ \emph{normalize} ฟังก์ชันคลื่น:
\begin{eqbox}{เงื่อนไขการ Normalize}
    \int_{-\infty}^\infty \abs{\Psi(x, t)}^2 \odif{x} = 1 \label{1normalize}
\end{eqbox}
โดยเราสามารถทำเช่นนี้ได้กับทุก ๆ ฟังก์ชันคลื่นโดยการคูณด้วยค่าคงที่เข้าไป (จริง ๆ แล้วยังมีฟังก์ชันที่ทำให้อินทิกรัลลู่ออกที่ไม่สามารถ normalize ได้ แต่เราจะสมมติว่าฟังก์ชันคลื่นเหล่านั้นไม่สามารถพบได้หรือ \emph{non-physical}) และเมื่อ normalize แล้ว การวิวัฒนาการของฟังก์ชันคลื่นจะไม่ทำให้อินทิกรัลใน (\ref{1normalize}) เปลี่ยนค่าซึ่งพิสูจน์ต่อไปนี้
\begin{eqbox}{บทตั้ง}
    \odv{}{t} \int_{-\infty}^\infty \abs{\Psi(x, t)}^2 \odif{x} = 0
\end{eqbox}
\begin{proof}
    เริมจากการหา
\end{proof}